\chapter{Video}
\section{Allgemeines}
\renewcommand{\kapitelautor}{Autor: Kerstin Schön}
Dieses Kapitel befasst sich mit der Planung, der Durchführung und der Bearbeitung der Videos. 

\subsection{Storyboard}
\renewcommand{\kapitelautor}{Autor: Kerstin Schön}
Um die korrekte Ausführung der Videos zu gewährleisten, wurden sogenannte Storyboards angefertigt. Diese wurden mit der Online Plattform "StoryboardThat" erstellt und schließlich als PDF - Format exportiert.
Im Storyboard wurden die Szenen bildlich dargestellt. Dies unterstützte das Team insofern im weiteren Verlauf, um grobe Fehler zu vermeiden.
Das Online - Tool ermöglichte es zwischen verschiedensten Szenen, Charaktären und Kategorien auszuwählen, wodurch einfach, verschiedenste Szenen erstellt werden konnten.
 
\subsection{Fragenkatalog}
\renewcommand{\kapitelautor}{Autor: Kerstin Schön}
Zusätzlich zum Storyboard wurden für das jeweilige Video die entsprechenden Fragen erstellt. Hierbei war es wichtig die Fragen an die Zielgruppe anzupassen. 

\subsection{Setting}
Beim Aufbau des Sets wurde zunächst auf das Storyboard referenziert, welche Situation man vorfindet. 

\subsection{Kamera}
%https://www.canon.at/for_home/product_finder/cameras/digital_slr/technologies_features/cmos_sensor.aspx
%https://www.itwissen.info/CMOS-Sensor-CMOS-sensor.html
Zum Drehen der Videos kam die Spiegelreflexkamera Canon EOS 70D zum Einsatz. Die Kamera Canon EOS 70D hat den CMOS Sensor verbaut, welcher im Gegensatz zu dem CCD (charge-coupled device) - Sensor einige Vorteile mit sich bringt. Vereinfacht gesagt, sind CMOS bzw. CCD - Sensoren "lichtempfindliche Bauteile, die das auf sie fallende Licht in Spannung umwandeln".  

\subsection{Licht}
Um den Schauplatz genügend auszuleuchten, verwendeten wir für alle vier Videos LED Scheinwerfer. -> softbox 

\subsection{Mikrofone}
%https://shuredeutschland.wordpress.com/2013/02/08/mikrofongrundlagen-richtcharakteristik/
%http://www.shure.at/supportdownload/tipps_grundlagen/mikrofone/mikrofone-richtcharakteristik
%https://www.delamar.de/mikrofon/richtcharakteristik-mikrofon-22647/
Bevor man sich für ein Mikrofon entscheidet, müssen einige Dinge beachtet werden, um das optimalste Ergebnis zu erreichen. Es gibt Mikrofone mit einer Kugelcharakteristik, die sich z.B. für Aufnahmen gut eignen, wenn der Schall von allen Seiten aufgenommen werden soll. Will man möglichst gerine Hintergrundgeräusche erzielen, eignet sich optimal ein Mikrofon mit einer Nierencharakteristik. \\
Bei der Richtcharakteristik eines Mikrofons kommt es grundsätzlich darauf an, aus welcher Richtung man den Schall aufnehmen möchte. Eine Kugelcharakteristik eignet sich dann z.B. gut, wenn man mehrere Leute


verwendung von zoom bzw. einmal mit handy vorteile und nachteile 

\section{Interview mit dem Abteilungsvorstand}
\renewcommand{\kapitelautor}{Autor: Kerstin Schön}
\subsection{Idee}
\renewcommand{\kapitelautor}{Autor: Kerstin Schön}
\section{Tag der offenen Tür}
\renewcommand{\kapitelautor}{Autor: Kerstin Schön}
\subsection{Idee}
\renewcommand{\kapitelautor}{Autor: Kerstin Schön}
\section{Video mit einem Absolvent}
\renewcommand{\kapitelautor}{Autor: Kerstin Schön}
\subsection{Idee}
\renewcommand{\kapitelautor}{Autor: Kerstin Schön}

\section{Schnitt}
\renewcommand{\kapitelautor}{Autor: Kerstin Schön}

\section{Herausforderungen}
\renewcommand{\kapitelautor}{Autor: Kerstin Schön}
\subsection{Tiefenschärfe}
\renewcommand{\kapitelautor}{Autor: Kerstin Schön}
\subsection{Farbkorrektur}
\renewcommand{\kapitelautor}{Autor: Kerstin Schön}
\subsection{Audio}
\renewcommand{\kapitelautor}{Autor: Kerstin Schön}
\subsection{Vertonung}
\renewcommand{\kapitelautor}{Autor: Kerstin Schön}

\section{Videoexport}
\renewcommand{\kapitelautor}{Autor: Kerstin Schön}
\subsection{Browserkompatibilität}
\renewcommand{\kapitelautor}{Autor: Kerstin Schön}
