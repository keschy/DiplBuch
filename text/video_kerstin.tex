\chapter{Video}
\section{Allgemeines}
\renewcommand{\kapitelautor}{Autor: Kerstin Schön}
Dieses Kapitel befasst sich mit der Planung, der Durchführung und der Bearbeitung der Videos. 

\subsection{Storyboard}
\renewcommand{\kapitelautor}{Autor: Kerstin Schön}
Ein Storyboard ist eine visualiserte Veranschaulichung von dem Konzept, das man sich zuvor überlegt hat.\footnote{\label{foot:1}https://www.e-teaching.org/didaktik/konzeption/inhalte/storyboard}Um die korrekte Ausführung der Videos zu gewährleisten, wurden sogenannte Storyboards angefertigt. Diese wurden mit der Online Plattform Storyboard That\footnote{\label{foot:2}"https://www.storyboardthat.com/"} erstellt und schließlich als PDF - Format exportiert.
Im Storyboard wurden die Szenen bildlich dargestellt, was das Team im weiteren Verlauf unterstützte, da es dadurch grobe Fehler vermeiden konnte.
Das Online - Tool ermöglichte es zwischen verschiedensten Szenen, Charakteren und Kategorien auszuwählen, wodurch vereinfacht, verschiedenste Szenen dargestellt werden können.
 
\subsection{Fragenkatalog}
\renewcommand{\kapitelautor}{Autor: Kerstin Schön}
Zusätzlich zum Storyboard wurden, für das jeweilige Video, die entsprechenden Fragen erstellt. Hierbei war es wichtig, die Fragen an die Zielgruppe anzupassen, da Jugendliche Fragen anders erfassen, als Erwachsene. Weiters muss man bei Jugendlichen darauf achten, dass man sie mit bestimmten Fragen nicht überfordert, und trotzdem noch die korrete Antwort bekommt.

\subsection{Setting}
\renewcommand{\kapitelautor}{Autor: Kerstin Schön}
Beim Aufbau des Sets wurde zunächst auf das Storyboard referenziert. Hierbei wurde auch die Location berücksichtigt, da für  den Aufbau eines Videos mittels Greenscreen ein anderes Set von Nöten war, als bei einem Video am Gang in der Schule. Weiters musste überlegt werden, wie die Kameras und Lichter platziert werden müssen. Bevor es zum eigentlichen Dreh kam, wurden die jeweiligen Drehs im Vorhinein ausreichend getestet, um später beim wirklichen Dreh Fehler zu vermeiden. Weiters wurden zusätzlich verschiedene Mikrofone getestet, um für die jeweilige Situation das optimale Mikrofon zu verwenden.

\subsection{Kamera}
\renewcommand{\kapitelautor}{Autor: Kerstin Schön}
Zum Drehen der Videos kam die Spiegelreflexkamera Canon EOS 70D zum Einsatz. Die Kamera Canon EOS 70D hat einen CMOS Sensor verbaut, welcher im Gegensatz zu einem CCD (charge-coupled device) - Sensor einige Vorteile mit sich bringt. Vereinfacht gesagt, sind CMOS beziehungsweise CCD - Sensoren \textit{"lichtempfindliche Bauteile, die das auf sie fallende Licht in Spannung umwandeln."}\footnote{\label{foot:3}https://www.itwissen.info/CMOS-Sensor-CMOS-sensor.html}
Der CMOS - Sensor in der Kamera sorgt für einen hohen Dynamikbereich, das heißt die Detailinformation und die Farbnuancen der Bilder bzw. der Videos sind ziemlich präzise, und haben ein geringes Bildrauschen. Weiters bewerkstelligt der CMOS - Sensor eine verlängerte Akkulaufzeit, was durch den vereinfachten Ladungstransfer ermöglicht wird.\footnote{\label{foot:4}http://bit.ly/2FcqnNA} Dies war deswegen sehr esenziell, da das Drehen der Videos bis zu 30 Minuten dauerte.

\subsection{Licht}
\renewcommand{\kapitelautor}{Autor: Kerstin Schön}
Um den Schauplatz genügend auszuleuchten, verwendeten wir für alle vier Videos LED Scheinwerfer. Um keine harten Schatten zu erzeugen, wurden noch zusätzlich sogenannte Softboxen verwendet. Eine Softbox ist im Grunde eine Hülle, bei der die Innenseite silber ist, und welche dadurch das Licht wie ein Reflektor nach vorne reflektiert. Anschließend geht das Licht durch den sogenannten "Front - Diffuser", was ein lichtdurchlässiges Gewebe ist, der das Licht streut, und es so diffuser erscheinen lässt. Durch die große Fläche der Softbox, wird das Licht weicher, das bedeutet, die Schatten werden diffuser. Dadurch wird auch die Helligkeit reduziert.\footnote{\label{foot:4}https://abenteuerdslrfotografie.de/softbox-post/}

\subsection{Mikrofone}
\renewcommand{\kapitelautor}{Autor: Kerstin Schön}
Bevor man sich für ein Mikrofon entscheidet, müssen einige Dinge beachtet werden, um ein optimales Ergebnis zu erreichen. Es gibt Mikrofone mit einer Kugelcharakteristik, die sich z.B. für Aufnahmen gut eignen, wenn der Schall von allen Seiten aufgenommen werden soll. Will man möglichst geringe Hintergrundgeräusche erzielen, eignet sich optimal ein Mikrofon mit einer Nierencharakteristik. 
Bei der Richtcharakteristik eines Mikrofons kommt es grundsätzlich darauf an, aus welcher Richtung man den Schall aufnehmen möchte. Eine Kugelcharakteristik eignet sich dann gut, wenn man beispielsweise mehrere Leute gleichzeitig aufnehmen will.\footnote{\label{foot:5}https://www.delamar.de/mikrofon/richtcharakteristik-mikrofon-22647/}

\section{Interview mit dem Abteilungsvorstand}
\renewcommand{\kapitelautor}{Autor: Kerstin Schön}
\subsection{Idee}
Die Idee des Interviews mit dem Abteilungsvorstand Dr. Hager war es, allgemeine Fragen über die Schule zu beantworten. In dem Video wurden Fragen, wie zum Beispiel "Muss sich ein Schüler schon nach der ersten Klasse für eine Abteilung entscheiden?" beantwortet, um etwaige Fragen im Vorraus zu klären.
 
\section{Tag der offenen Tür}
\renewcommand{\kapitelautor}{Autor: Kerstin Schön}
\subsection{Idee}
Das Konzept des Tag der offenen Tür Videos war es, Interessenten zu interviewen, um den ersten Eindruck der Jugendlichen festzuhalten. 

\section{Video mit einem Absolvent}
\renewcommand{\kapitelautor}{Autor: Kerstin Schön}
\subsection{Idee}
Die Intention hinter dem Video, mit dem Absolventen und einem Schüler aus der 2. Klasse, war es interessierten Schülerinnen und Schülern zu zeigen, dass man mit wenig bis keinem technischen Wissen in diese Schule kommt, aber dafür mit einem großen Wissen aus der Schule geht. Die meisten Schülerinnen und Schüler spezialisieren sich auf ein Themengebiet, wie z.B. Fotografie und Video oder Programmieren und Webdesign. Da die Medientechnik so ein enorm großer Bereich ist, ist es fast unmöglich, über alles und jenes Bescheid zu wissen. Dennoch soll das Video veranschaulichen, dass man trotz alldem die Grundlagen der anderen Themengebiete beherrscht.

\section{Schnitt}


\section{Auswahl des Videoprogrammes}
\renewcommand{\kapitelautor}{Autor: Kerstin Schön}

\subsection{Premiere Pro}

\section{Herausforderungen}
\renewcommand{\kapitelautor}{Autor: Kerstin Schön}
\subsection{Interview mit dem Abteilungsvorstand}
Da bei dem Drehen des Videos mit dem Abteilungsvorstand die nötige Tiefenschärfe gefehlt hat, war es notwendig diese in Premiere CS6 manuell nachzubearbeiten. Dies war deswegen nötig, da bei fehlender Tiefenschärfe, der User vom Hintergrund zu sehr abgelenkt wäre, und so das eigentliche Video im Hintergrund stehen würde.
In Premiere CS6 ist es mittels der bewegten Maske möglich, eine Maske zu setzen, wo die Tiefenschärfe schlussendlich gesetzt werden soll, die man entweder händisch, mit einer Kreis Form oder einem Rechteck erstellen kann. Hat man die Maske gezeichnet, kann man mittels des Effektes "Gaußscher Weichzeichner" einen Tiefenschärfe Effekt erzeugen.
\subsection{Tag der offenen Tür Video}
\subsection{Video mit einem Absolvent}

\subsection{Tiefenschärfe}
\subsection{Farbkorrektur}
\subsection{Audio}
\subsection{Vertonung}

\section{Videoexport}
\subsection{Browserkompatibilität}
\renewcommand{\kapitelautor}{Autor: Kerstin Schön}
Erklärung H.264 und WebM + Browserkompilationen, etc.
Um zu gewährleisten, dass jeder User Zugang zu den Videos hat, wurden die Videos nicht nur im H.264 Codec gerendert, sondern auch im Web.m Codec. Heutzutage, unterstützen alle neuen Browser Versionen den H.264 Codec, jedoch wird nicht von jedem User vorausgesetzt, dass dieser auch die neuesten Browser Versionen installiert hat. Da frühere Mozilla Firefox Browser Versionen noch keinen H.264 Codec unterstützen, ist es wichtig die Videos auch in Web.m zu rendern. 