\chapter{Design}
\renewcommand{\kapitelautor}{Autor: Niklas Kienreich}

\section{Hintergedanken}
Genau, wie bei allen anderen Aspekten, dieser Diplomarbeit, war der Hintergedanke des Designs, es einladend, fröhlich und angemessen der Zielgruppe zu gestalten.

\section{Richtlinien}
\section{Corporate Design}
\subsection{Farben}
Bei den Farben wurde zum einen, ein helles Grün, zum anderen ein knalliges Orange gewählt. Als Akzentfarbe wurde ein helles Grau in das Corporate Design aufgenommen.

\subsubsection{Grün}
Wenn etwas ergrünt, ergibt sich Hoffnung auf neues Leben. Diese Farbe ist sehr naturbelassen und die Farbpsychologie lehrt, dass Grün Lebensfreude und Jugend vermittelt. Grün kann zwar auch Gift und somit eine Warnung signalisieren, durch den Kontext der Webseite ist dies aber unwahrscheinlich. \footnote{\label{foot:1} http://www.grafixerin.com/bilder/Farbpsychologie.pdf [Zugriff: 17.03.2018]}

\subsubsection{Orange}
Bei Orange handelt es sich um eine sehr auffällige Farbe. Sie ist eine warme, energiegeladene Farbe und kann aufregend, bis sogar spaßig auf einen Menschen wirken. Durch ihre Auffälligkeit kann sie aufdringlich oder extrovertiert wirken. Orange kann auch billig wirken, da der Mensch kaum etwas Hochwertiges kennt, dass diese Farbe besitzt. \footnote{\label{foot:1} http://www.grafixerin.com/bilder/Farbpsychologie.pdf [Zugriff: 17.03.2018]}

\subsubsection{Grau}
\subsection{Typografie}
\subsubsection{Parisien Night Oblique}
Diese Schrift wurde wegen ihrer dynamischen Linienführung gewählt, welche an das Logo erinnert.
\\
Verwendung findet die Schrift in den Überschriften der Website.

\subsubsection{Calibri}
Calibri ist eine serifenlose Schrift. Sie ist dadurch auf Bildschirmen mit geringerer Auflösung leicht zu lesen. Sie wirkt durch ihre abgerundeten Ecken freundlicher, als manch andere Schriftart. \footnote{\label{foot:2} https://www.typografie.info/3/Schriften/fonts.html/calibri-r61/ [Zugriff: 18.03.2018]}
\\
Calibri wird in den Texten der Website und in den Textslides der Videos verwendet.

\subsection{Adobe Illustrator}
\subsubsection{Vectorgrafiken}
Vektorgrafiken kann man beliebig vergrößern und verkleinern, ohne dass sie an Auflösung, Qualität oder allgemein Daten verlieren. Das liegt daran, dass Vektorgrafiken aus Linien und Kurven bestehen. Diese zusammen bilden sogenannte Pfade. Was diese Pfade so speziell macht und die verlustfreie Skalierung ermöglicht, ist dass sie durch mathematische Funktionen beschrieben werden. Das heißt, dass das Bild auf jeder erdenklichen Größe neu berechnet und gerendert werden kann.
\footnote{\label{foot:1} https://helpx.adobe.com/illustrator/using/drawing-basics.html [Zugriff: 18.03.2018]}

\subsection{Logo}
\subsubsection{Logo-Arten}
Bei Logos unterscheidet man zwischen vier verschiedenen Arten. \footnote{\label{foot:3} http://www.webmasterpro.de/design/article/logodesign-arten-von-logos.html [Zugriff: 18.03.2018]} \footnote{\label{foot:4} http://www.sanseg.de/logodesign/zeichenmarken/ [Zugriff: 18.03.2018]}
\\
\\
Zum einen gibt es sogenannte Bildmarken. Hierbei handelt es sich um eine einfache Abbildung oder Illustration. Empfehlenswert ist solch ein Logo jedoch eher für größere Firmen und Unternehmen, da der Betrachter beim Erblicken des Logos, die Verbindung mit jenem Unternehmen herstellen können soll. Dies ist bei unbekannten Marken unwahrscheinlich und somit kontraproduktiv. Wenn ein Unternehmen auf eine Bildmarke umsteigt, gibt es oft eine Einführungsphase, in der das neue Logo anfangs mit Beschriftung gezeigt wird.
\\
\\
Eine weitere Logo-Art ist die Wortmarke. Sie zeigt einfach den Namen, des Unternehmens. Aber auch hier muss man sich Gedanken machen, wie das Logo auf den Betrachter wirkt. Neben der Farbe des Schriftzugs, sind typografische Aspekte auch wichtig. Darauf wird später in der Umsetzung des Logos genauer eingegangen.
\\
\\
Zeichenmarken sind den Wortmarken sehr ähnlich, denn sie beruhen auch auf der Typografie. Der Unterschied liegt darin, dass es sich hier nicht um Worte im Logo handelt, sondern um Abkürzungen, einzelne Buchstaben oder Zahlen.
\\
\\
Bei Wort-Bild-Marken, oder auch Kombinationen, handelt es sich, wie der Name schon verrät, um eine Kombination aus Bildmarke und entweder Wortmarke, oder Zeichenmarke. Bei dieser Art des Logos handelt es sich um die weit verbreitetste Variante. Sie ist auch am besten für kleine und mittelgroße Unternehmen geeignet.
\\
\\
Auch Keyvisuals können genutzt werden. Sie sind nicht an das Logo gebunden und können somit davon abweichen und eigene Charakteristiken veranschaulichen. Ein Keyvisual ist dann gut, wenn auf ihm der Name des Unternehmens nicht vertreten ist, der Betrachter es aber trotzdem ohne viel nachzudenken zuordnen kann.

\subsubsection{Zweck}
Das Wort Logo stammt von dem griechischen Wort „logos“ ab. Es lässt sich in „Wort“, aber auch „Rede“ und „Sinn“ übersetzen. Hierbei möchte man nun meinen, dass es sich bei einem Logo um ein Wort handeln muss. Da sich das Wort aber von logos ableitet und nicht eins zu eins das Selbe bedeutet, ist ein gewisser Spielraum gegeben. In der Praxis verstehen wir unter einem Logo meist ein sogenanntes Signet, welches sich vom latinischen Wort „signum“, also Zeichen, ableitet. Dieses kann, wie bereits erläutert wurde, in Kombination mit einem Wort verwendet werden. Die Bedeutung von „logos“ ist aber nicht unwichtig. Das Logo hat den Sinn als Kennzeichen zu fungieren. Dabei ist es egal ob es an ein Unternehmen, eine Dienstleistung oder ein Produkt gebunden ist. Es ist für alle diese Fälle einsetzbar. Ein weiterer Zweck von Logos ist die Orientierungs- und Entscheidungshilfe. Die Zielgruppe kann an diesen nämlich die Unternehmen, die hinter den Produkten oder Dienstleistungen stehen eindeutig identifizieren.
\\ 
Natürlich soll das Logo in seiner Form und Farbe, also seinem Erscheinungsbild, möglichst einzigartig und unverwechselbar sein. Das hat zwei große Gründe:
\begin(itemize)
\item Die Sinnhaftigkeit würde ansonsten verloren gehen.  Das kritische Stichwort hier ist „unverwechselbar“, denn wenn das Logo nicht einzigartig ist, ist es auch verwechselbar. Sobald ein Logo verwechselt werden kann, erfüllt es nicht mehr optimal seinen Zweck, die Bezugspersonen an das Unternehmen zu erinnern. Auf jeden Fall nicht spezifisch an das eigene Unternehmen, was sehr kontraproduktiv wäre.
\\
\item Es kann rechtliche Folgen nach sich ziehen. Hierbei ist das Wort „einzigartig“ das wichtigere. Wenn bei dem Logo zu hohe Verwechslungsgefahr besteht, beziehungsweise es eins zu eins das Selbe ist, riskiert man eine Klage, eines anderen Unternehmens, oder gar des Urhebers des Logos. Regionale Kleinunternehmen werden in den Klagen von Global Playern nicht ausgeschlossen, also ist hier größte Vorsicht geboten.
!!!!!!!!!!!!!!!!!!!!!!!!!!!!!!!!!!!!!!!!!!!!!!!   Quelle fehlt noch   !!!!!!!!!!!!!!!!!!!!!!!!!!!!!!!!!!!!!!!!!!!!!!!
\end(itemize)
\subsubsection{Symbolik}
\subsubsection{Umsetzung}
Das Logo von Insight wurde nach dem Brainstorming des Teams, zuerst in Form von Skizzen für das Signet in Adobe Photoshop CS6 gezeichnet. Hierbei wurde ein Grafiktablett verwendet, da es eine einfachere, schönere und vor allem genauere Linienführung, als eine Maus gewährleistet.
Nachdem ein Entwurf ausgesucht wurde, wurde die Skizze exportiert und das Bild in eine Ebene, im Adobe Illustrator CS6 geladen, einem Vektorgrafikprogramm. Dieses Programm wurde genutzt, um eine Vektorgrafik, des Logos zu erstellen. Die Ebene hatte eine Transparenz, oder auch Opazität, von ungefähr 33%. So lenkt die Skizze nicht mehr von dem Ergebnis ab, kann aber doch noch gut auf dem weißen Hintergrund gesehen werden. Auf den darüber liegenden Ebenen wurde das Zeichenstift-Werkzeug genutzt um die Skizze als Vektorgrafik nachzuzeichnen. Das Pinsel-Werkzeug wäre mit dem Grafiktablett zwar leicht zu bedienen gewesen, hätte aber nicht genau das Ergebnis erbracht, was das andere Werkzeug gewährleisten konnte.

\begin(itemize)
\item Das Zeichenstift-Werkzeug (https://helpx.adobe.com/de/illustrator/using/drawing-pen-pencil-or-flare.html	http://www.mathepedia.de/Bezierkurven.html) Mit dem ersten Klick, mit dem Zeichenstift-Werkzeug, legt man den ersten Ankerpunkt der Form fest. Danach kann man durch Klicken, immer neue Ankerpunkte setzen. Zwischen jedem Ankerpunkt wird eine gerade Linie gezogen. Wenn man nun eine Kurve ziehen möchte, klickt man, lässt dabei gedrückt, und zieht mit dem Cursor über den Bildschirm. Hierbei wird eine Bézierkurve erstellt. Diese wird durch Parameter beschrieben. Eine Bézierkurve n-ten Grades benötigt n+1 Punkte um beschrieben zu werden. Also wird eine einfache Kurve, die mit dem Zeichenstift-Werkzeug erstellt wird, durch den Anfangs- und Endankerpunkt beschrieben und die Endpunkte der Tangenten, die von diesen zwei Punkten ausgehen.
Um den Pfad zu beenden, kann man zwei verschiedene Varianten wählen. Wenn man die Form schließen möchte, muss man einfach nur die letzte Linie oder Kurve mit dem ersten Ankerpunkt verbinden. Sollte man den Pfad aber geöffnet lassen wollen, kann man mit gedrückt gehaltener Strg-Taste mit dem Cursor außerhalb der Form klicken. Noch einfacher ist es, bei noch nicht geschlossener Form einfach ein anderes Werkzeug auszuwählen.
\item Das Pinsel-Werkzeug (https://helpx.adobe.com/de/illustrator/using/brushes.html) Bei dem Pinsel-Werkzeug ist es, ähnlich zu den Pinseln in Photoshop, nur notwendig mit freier Hand eine Linie, Kurve oder Folge der beiden genannten zu zeichnen. Das Programm berechnet daraufhin von selbst die notwendigen Parameter und vektorisiert den Pfad. Um hier eine geschlossene Form zu zeichnen, muss nur die Alt-Taste gedrückt gehalten werden. Trotz Einstellungsmöglichkeiten, wie Genauigkeit, oder Glättung, des Pinsels, fiel die Auswahl für das Nachzeichnen der Logoskizze nicht auf ihn, da die Skizze bereits sehr genaue Proportionen hatte, die mit freier Hand schwer zu treffen waren.
\end(itemize)

Für die runden Elemente, wie Iris, Pupille und die Reflexionen, wurde das Ellipse-Werkzeug verwendet.

\subsection{Visitenkarten}
\subsection{Designelemente}