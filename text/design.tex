\chapter{Design}
\renewcommand{\kapitelautor}{Autor: Niklas Kienreich}

\section{Hintergedanken}
Genau, wie bei allen anderen Aspekten, dieser Diplomarbeit, war der Hintergedanke des Designs, es einladend, fröhlich und angemessen der Zielgruppe zu gestalten.

\section{Richtlinien}
\section{Corporate Design}
\subsection{Farben}
Bei den Farben wurde zum einen, ein helles Grün, zum anderen ein knalliges Orange gewählt. Als Akzentfarbe wurde ein helles Grau in das Corporate Design aufgenommen.

\subsubsection{Grün}
Wenn etwas ergrünt, ergibt sich Hoffnung auf neues Leben. Diese Farbe ist sehr naturbelassen und die Farbpsychologie lehrt, dass Grün Lebensfreude und Jugend vermittelt. Grün kann zwar auch Gift und somit eine Warnung signalisieren, durch den Kontext der Webseite ist dies aber unwahrscheinlich. \footnote{\label{foot:1} http://www.grafixerin.com/bilder/Farbpsychologie.pdf [Zugriff: 17.03.2018]}

\subsubsection{Orange}
Bei Orange handelt es sich um eine sehr auffällige Farbe. Sie ist eine warme, energiegeladene Farbe und kann aufregend, bis sogar spaßig auf einen Menschen wirken. Durch ihre Auffälligkeit kann sie aufdringlich oder extrovertiert wirken. Orange kann auch billig wirken, da der Mensch kaum etwas Hochwertiges kennt, dass diese Farbe besitzt. \footnote{\label{foot:1} http://www.grafixerin.com/bilder/Farbpsychologie.pdf [Zugriff: 17.03.2018]}

\subsubsection{Grau}
\subsection{Typografie}
\subsubsection{Parisien Night Oblique}
Diese Schrift wurde wegen ihrer dynamischen Linienführung gewählt, welche an das Logo erinnert.
\\
Verwendung findet die Schrift in den Überschriften der Website.

\subsubsection{Calibri}
Calibri ist eine serifenlose Schrift. Sie ist dadurch auf Bildschirmen mit geringerer Auflösung leicht zu lesen. Sie wirkt durch ihre abgerundeten Ecken freundlicher, als manch andere Schriftart. \footnote{\label{foot:2} https://www.typografie.info/3/Schriften/fonts.html/calibri-r61/ [Zugriff: 18.03.2018]}
\\
Calibri wird in den Texten der Website und in den Textslides der Videos verwendet.

\subsection{Adobe Illustrator}
\subsubsection{Vectorgrafiken}

\subsection{Logo}
\subsubsection{Logo-Arten}
Bei Logos unterscheidet man zwischen vier verschiedenen Arten. \footnote{\label{foot:3} http://www.webmasterpro.de/design/article/logodesign-arten-von-logos.html [Zugriff: 18.03.2018]} \footnote{\label{foot:4} http://www.sanseg.de/logodesign/zeichenmarken/ [Zugriff: 18.03.2018]}
\\
\\
Zum einen gibt es sogenannte Bildmarken. Hierbei handelt es sich um eine einfache Abbildung oder Illustration. Empfehlenswert ist solch ein Logo jedoch eher für größere Firmen und Unternehmen, da der Betrachter beim Erblicken des Logos, die Verbindung mit jenem Unternehmen herstellen können soll. Dies ist bei unbekannten Marken unwahrscheinlich und somit kontraproduktiv. Wenn ein Unternehmen auf eine Bildmarke umsteigt, gibt es oft eine Einführungsphase, in der das neue Logo anfangs mit Beschriftung gezeigt wird.
\\
\\
Eine weitere Logo-Art ist die Wortmarke. Sie zeigt einfach den Namen, des Unternehmens. Aber auch hier muss man sich Gedanken machen, wie das Logo auf den Betrachter wirkt. Neben der Farbe des Schriftzugs, sind typografische Aspekte auch wichtig. Darauf wird später in der Umsetzung des Logos genauer eingegangen.
\\
\\
Zeichenmarken sind den Wortmarken sehr ähnlich, denn sie beruhen auch auf der Typografie. Der Unterschied liegt darin, dass es sich hier nicht um Worte im Logo handelt, sondern um Abkürzungen, einzelne Buchstaben oder Zahlen.
\\
\\
Bei Wort-Bild-Marken, oder auch Kombinationen, handelt es sich, wie der Name schon verrät, um eine Kombination aus Bildmarke und entweder Wortmarke, oder Zeichenmarke. Bei dieser Art des Logos handelt es sich um die weit verbreitetste Variante. Sie ist auch am besten für kleine und mittelgroße Unternehmen geeignet.
\\
\\
Auch Keyvisuals können genutzt werden. Sie sind nicht an das Logo gebunden und können somit davon abweichen und eigene Charakteristiken veranschaulichen. Ein Keyvisual ist dann gut, wenn auf ihm der Name des Unternehmens nicht vertreten ist, der Betrachter es aber trotzdem ohne viel nachzudenken zuordnen kann.

\subsubsection{Zweck}
\subsubsection{Symbolik}
\subsubsection{Umsetzung}
\subsection{Visitenkarten}
\subsection{Designelemente}