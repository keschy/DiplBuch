\chapter{Design}
\renewcommand{\kapitelautor}{Autor: Niklas Kienreich}

\section{Hintergedanken}
\renewcommand{\kapitelautor}{Autor: Niklas Kienreich}
Genau wie bei allen anderen Aspekten, diese Diplomarbeit, war der Hintergedanke des Designs, es einladend, fröhlich und angemessen der Zielgruppe zu gestalten.

\section{Richtlinien}
\section{Corporate Design}
\subsection{Farben}
\renewcommand{\kapitelautor}{Autor: Niklas Kienreich}
Bei den Farben wurde zum einen ein helles Grün, zum anderen ein knalliges Orange gewählt. Als Akzentfarbe wurde noch ein helles Grau in das Corporate Design aufgenommen.

\subsubsection{Grün}
\renewcommand{\kapitelautor}{Autor: Niklas Kienreich}
Wenn etwas ergrünt, ergibt sich Hoffnung auf neues Leben. Diese Farbe ist sehr naturbelassen und die Farbpsychologie lehrt, dass Grün Lebensfreude und Jugend vermittelt. Grün kann zwar auch Gift und somit eine Warnung signalisieren, durch den Kontext der Webseite ist dies aber unwahrscheinlich. \footnote{\label{foot:1} http://www.grafixerin.com/bilder/Farbpsychologie.pdf [Zugriff: 17.03.2018]}

\subsubsection{Orange}
\renewcommand{\kapitelautor}{Autor: Niklas Kienreich}
Bei Orange handelt es sich um eine sehr auffällige Farbe. Sie ist eine warme, energiegeladene Farbe und kann aufregend bis sogar spaßig auf einen Menschen wirken. Durch ihre Auffälligkeit kann sie aufdringlich oder extrovertiert wirken. Orange kann auch billig wirken, da der Mensch kaum etwas Hochwertiges kennt, dass diese Farbe besitzt. \footnote{\label{foot:1} http://www.grafixerin.com/bilder/Farbpsychologie.pdf [Zugriff: 17.03.2018]}

\subsubsection{Grau}
\subsection{Typografie}
\subsubsection{Parisien Night Oblique}
\renewcommand{\kapitelautor}{Autor: Niklas Kienreich}
Diese Schrift wurde wegen ihrer dynamischen Linienführung gewählt, welche an das Logo erinnert.
\\
Verwendung findet die Schrift in den Überschriften der Website.


\subsubsection{Calibri}
\renewcommand{\kapitelautor}{Autor: Niklas Kienreich}
Calibri ist eine serifenlose Schrift. Sie ist dadurch auf Bildschirmen mit geringerer Auflösung leicht zu lesen. Sie wirkt durch ihre abgerundeten Ecken freundlicher, als manch andere Schriftart. \footnote{\label{foot:2} https://www.typografie.info/3/Schriften/fonts.html/calibri-r61/ [Zugriff: 18.03.2018]}
\\
Calibri wird in den Texten der Website und in den Textslides, der Videos verwendet.


\subsection{Logo}
\subsection{Visitenkarten}
\subsection{Designelemente}