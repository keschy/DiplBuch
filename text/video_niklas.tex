\section{Interview mit einem Fachmann}
\subsection{Idee}
\renewcommand{\kapitelautor}{Autor: Niklas Kienreich}
\subsection{Ziel}
\renewcommand{\kapitelautor}{Autor: Niklas Kienreich}
\subsection{Dreh}
\renewcommand{\kapitelautor}{Autor: Niklas Kienreich}
\subsection{Schnitt}
\renewcommand{\kapitelautor}{Autor: Niklas Kienreich}
\subsection{Green Screen}
\renewcommand{\kapitelautor}{Autor: Niklas Kienreich}
\subsection{Farbkorrektur}
\renewcommand{\kapitelautor}{Autor: Niklas Kienreich}


\chapter{Animationsvideo}
\renewcommand{\kapitelautor}{Autor: Niklas Kienreich}
\section{Idee}
\renewcommand{\kapitelautor}{Autor: Niklas Kienreich}
Das Animationsvideo sollte als Eyecatcher für die Zielgruppe dienen. Der Gedanke war, Allgemeines über die HTL auf möglichst witzige, ansprechende Weise zu vermitteln. Die Animation schafft das besser, als die restlichen Videos, da man sich bei den Interviews auf eine lockere, aber doch ernste, zielführende Gesprächsführung verlassen hat. Eine Frage die man sich nun aber stellen musste war, wie man das Video animiert. Neben vier Interview Videos, die nicht nur gedreht, sondern auch geschnitten und farbkorrigiert werden mussten und der Website, war für so ein kleines Team von drei Personen nicht allzu viel Zeit eingeplant. Nach überraschend kurzer Recherche, war nicht nur eine einfache, sondern auch ansprechende Lösung gefunden. In den letzten Jahren haben immer mehr Kanäle auf Youtube bei unserer Zielgruppe großes Interesse geweckt. Laut Youtube exestiert einer der beliebteren Kanäle seit dem 30.08.2014 und hat seitdem fast 6 Millionen Abonnenten und 866 Millionen Aufrufe gesammelt. Ihre Videos sind kurze Animationen im vereinfachten Stil. Also keine flüssigen Bewegungen, sondern mehr sprunghafte Frames mit einem Voice-over. Es erinnert an eine digitale Version eines so genannten „Draw my life“.
\section{Ziel}
\renewcommand{\kapitelautor}{Autor: Niklas Kienreich}
Das Ziel war es, mit dem zwar kürzesten, aber unterhaltsamsten Video, die wichtigsten, grundlegenden Informationen zu Höheren Technische Bundeslehranstalten zu bieten. Die detaillierten Informationen zu der Schule selbst, sind im Interviewvideo mit dem Abteilungsvorstand zu finden. Die User sollten nach den beiden Videos zuerst wissen, ob sie an solch einer Schule überhaupt Interesse haben. Ein Wissen das essenziell ist, bevor man sich auf die Frage stürzt, ob die Medientechnik, oder gar irgendein anderer technischer Fachbereich für einen geeignet ist.
\section{Drehbuch}
\renewcommand{\kapitelautor}{Autor: Niklas Kienreich}
\section{Audioaufnahme}
\renewcommand{\kapitelautor}{Autor: Niklas Kienreich}
\section{Schnitt}
\renewcommand{\kapitelautor}{Autor: Niklas Kienreich}
\section{Umsetzung}
\renewcommand{\kapitelautor}{Autor: Niklas Kienreich}
