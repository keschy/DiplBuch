\chapter{Das Projekt}

In diesem Kapitel, wird die Idee und Planung des Projektes erläutert. 

\section{Die Idee}
Am Tag der offenen Tür werden jungen Schülern gezeigt, was die Medientechnik beinhaltet. Dabei werden nur die Bereiche Video, Audio und Fotografie demonstriert. Bereiche wie Webdesign, Softwaretechnik und Webtechnologie werden gänzlich ausgelassen. Dadurch, dass die Besucher denken, dass es nur diese multimedialen Bereiche gibt, melden sie sich an und merken schnell, dass man sich erst ab der 4. Klasse auf diese Bereiche spezialisiert. Um dies zu verhindern, wurden Informationsvideos gedreht, die auf einer Webseite zur Verfügung stehen. Es wurden insgesamt fünf Videos auf die Webseite gestellt. Zwei davon sind Interviews mit dem Abteilungsvorstand der HTL Rennweg und dem Medientechniklehrer Mag. Roman Jerabek. Die übrigen Videos sorgen für den „Fun-Factor“ und sind humorvoller als die Interviews. Die Webseite beinhaltet interaktive Elemente, welche zeigen, was mit der Medientechnik gemacht werden kann. Durch diese Mittel soll gezeigt werden, was in den fünf Jahren beigebracht wird.

\section{Die Planung}
Die Projektplanung ist die wichtigste Phase eines Projektes und erfolgt, nach der Definierung der Projektziele.\cite{pmplanung} Weiters wird die Projektmanagementmethode und alle, für diese Methode wichtigen, Tools, welche verwendet wurden, erklärt. Im Rahmen dieser Diplomarbeit, wurde als Erstes ein Kick-Off-Meeting mit den Projektbetreuern organisiert, bei der die Durchführung besprochen wurde. Anschließend wurden die zu erledigenden Dokumente, die Umwelt- bzw. Umfeldanalyse und die Risikoanalyse, festgelegt. Zuletzt wurde die Projektmanagementmethode festgelegt, welche für diese Diplomarbeit am besten geeignet ist. Nach dem Kick-Off-Meeting, wurde der Antrag, welcher die Projektziele, die Projektorganisation, die Umwelt- bzw. Umfeldanalyse, die Risikoanalyse, die Meilensteinliste, die Projektressourcen und externe Kooperationspartner, enthält geschrieben und dem Abteilungsvorstand der IT, Dr. Gerhard Hager, übergeben.

\section{Projektmanagement und Kommunikation}

Für diese Diplomarbeit entschied sich das Team dazu, die Projektmanagementmethode Scrum zu verwenden. Für diese Methode ist ein Tool, welches die Arbeitspakete enthält, notwendig. Hierbei verwendet das Team die Webseite taiga.io, welches mit der Webseite für Zeiterfassung, toggl.com, verwendet werden kann. Für die Dokumentenverwaltung wird Sharepoint verwendet, welche mit den HTL Rennweg Accounts verbunden sind. Dabei hat der Hauptbetreuer, sowie der Nebenbetreuer Zugriff darauf und kann alle Dokumente einsehen.

\subsection{Projektmanagementmethode} 

Die vom Projektteam gewählte Projektmanagementmethode ist Scrum. Scrum ist eine agile Projektmanagementmethode und wird seit den 90er Jahren in der Softwareentwicklung verwendet und findet dort seinen Ursprung. Es werden drei Rollen definiert, der Product Owner, das Entwicklungsteam und der Scrum Master. \cite{scrum}

\subsubsection{Product Backlog} 
Der Product Backlog wird vom Product Owner angelgt, gepflegt, geordnet und priorisiert und ist eine Sammlung von Anforderungen. Das Backlog enthält „User-Stories“, welche die Bedürfnisse der Nutzer widerspiegeln. Diese enthalten grobe Anforderungen, die für das Endergebnis wichtig sind und werden beim Sprint-Planning Meeting festgelegt. Die User-Stories müssen in kleinstmögliche Aufgaben aufgegliedert werden, damit das Team diese im Sprint abarbeiten kann. Ein Sprint dauert in der Regel ein bis vier Wochen, diese wird vom Scrum Master festgelegt. Weiters, spielt die Priorisierung eine wichtige Rolle, denn diese entscheidet darüber, was im nächsten Sprint abgearbeitet wird. Diese darf während dem Sprint nicht verändert werden. Allerdings werden sie in Absprache mit den Stakeholdern korrigiert und angepasst, sofern die Dringlichkeit und Notwendigkeit eine große Rolle spielen. Um einen Überblick über alle Aufgaben zu behalten, wird ein Sprint Backlog angelegt. \cite{productbacklog}

\subsubsection{Sprint Backlog} 
In einem Sprint Backlog befindet sich eine Liste von Aufgaben. Diese sind essentiell um die Anforderungen des Product Backlogs umzusetzen. Das Sprint Backlog entsteht aus der Sprint Planung, wobei der Aufwand einer Aufgabe in Stunden geschätzt wird. Die Aufgaben nimmt sich jedes Teammitglied selbst und nutzt den Sprint Backlog zur Verfolgung der Aufgaben. \cite{backlog}

\subsubsection{Sprint Planning} 
Im Sprint Planning werden die Anforderungen und Arbeitspakete, die im aktuellen Sprint umgesetzt werden sollen, besprochen. Der Product-Owner organisiert alles, das Entwicklungsteam, der Scrum-Master und optional der Stakeholder, nehmen teil.  Das Sprint Planning findet zu Beginn jedes Sprints statt und dauert vier bis acht Stunden. Aus dem Product-Backlog werden eine Anzahl an höchstpriorisierten Anforderungen ausgewählt und in ein Selected-Backlog übernommen. Diese sind nur Vorschläge, denn das Entwicklerteam entscheidet letztendlich, was im Sprint realisiert werden soll. Die Sprint-Planung teilt sich in zwei Phasen auf. In der ersten Phase, werden die Anforderungen detailliert besprochen. Dabei schätzt das Entwicklungsteam den Realisationsaufwand und legt fest, was im Sprint umgesetzt wird. In der zweiten Phase bespricht das Entwicklungsteam die Realisierung und zerlegt die Anforderungen in einzelne Tasks. Bei Fragen sollte der Product Owner erreichbar sein. \cite{sprintplanning}

\subsubsection{Daily Scrum} 
Ein Daily Scrum-Meeting ist ein kurzes, tägliches Meeting des Teams, welches ungefähr 15 Minuten dauert. Der Scrum Master nimmt teil und greift moderierend ein, wenn nötig. Beim Daily Scrum werden Statusupdates und Geschehnisse geschildert. \cite{dailyscrum}

\subsubsection{Sprint Review} 
Das Sprint Review-Meeting findet am Ende jedes Sprintes statt. Sie dient zur Einholung von Feedback von den Stakeholdern. Dabei wird das Produktinkrement vorgestellt, wobei der Stakeholder seine Meinung gibt und dadurch das Produkt im nächsten Sprint optimiert werden kann. \cite{sprintreview}

\subsection{Tool für Scrum}
Taiga ist ein Open Source Tool, welches für Scrum Projekte verwendet werden kann. Durch das schlichte und moderne Design, findet man sich schnell zurecht. Man bekommt die Möglichkeit geboten entweder ein Scrum oder Kanban Template zu benutzen. In dieser Diplomarbeit wurde das Scrum Template verwendet. Taiga bietet ein Interface für das Product und Sprint Backlog und dem dazugehörigen Taskboard, welche alle Tasks grafisch darstellt und sie mit einem Status versieht, um ersichtlich zu machen, in welcher Phase sich der Task befindet. Es gibt fünf Phasen: 
\begin{itemize}
\item Neu
\item In Arbeit
\item Bereit zum Testen
\item Fertig
\item Informationen werden benötigt
\end{itemize}
Die Tasks können in die jeweilige Phase verschoben werden und sobald sie sich in der Phase „Fertig“ befinden, aktualisiert sich der Projektfortschritt. Das gibt einen genauen Überblick über den Stand des Projektes. 

\subsection{Zeiterfassung}
Für die Zeiterfassung wurde toggl.com verwendet. Durch die Eingabe des Tasks und den Klick eines Buttons, wird die Zeit aufgezeichnet und in den Reports aufgelistet. Der Projektleiter kann die erfassten Zeiten jedes Teammitgliedes sehen und auch verändern. Es gibt zwei Möglichkeiten die Zeit zu erfassen:
\begin{itemize}
\item Manual Mode
\item Timer Mode
\end{itemize}
Im Manual Mode wird die Zeit händisch eingetragen, währenddessen im Timer Mode die Zeit, ähnlich wie bei einer Stoppuhr, automatisch läuft. Das führt zu einer genaueren Zeiterfassung. 

\subsection{Kommunikationsmittel}
In der Diplomarbeit wurde teamintern WhatsApp als Kommunikationsmittel verwendet. WhatsApp ist eine Instant-Messaging-App mit der Nachrichten in Echtzeit ausgetauscht werden können. Die Projektleitung erstellte zu diesem Zwecke eine Gruppe, in welcher alle Mitglieder eingeladen wurden. In dieser Gruppe wurden wichtige Ereignisse besprochen, sowie Fragen geklärt. 
Mit den Diplomarbeitsbetreuern wurde die Kommunikation durch E-Mail bevorzugt. 

\subsection{Dokumentenverwaltung durch Sharepoint}
Für die Dokumentenverwaltung, entschied sich das Team für Microsoft Sharepoint. Dieses Tool bietet eine Verbesserung der Teamproduktivität, sowie ein bequemes Verwalten von Dokumente durch eine kurze Einarbeitungszeit und einer schnellen und leichten Erstellung einer Zusammenarbeitsumgebung. Auf Sharepoint haben das gesamte Team und die Projektbetreuer Zugriff, um hochgeladene Dateien bewerten zu können.

