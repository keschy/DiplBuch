\chapter{Das Projekt}
\renewcommand{\kapitelautor}{Autor: Hatice Akyokus}
In diesem Kapitel wird die Planung des Projektes erläutert. Die Projektplanung ist die wichtigste Phase eines Projektes und erfolgt nach der Definierung der Projektziele.\footnote{\label{foot:2}vgl.https://www.rechnungswesen-verstehen.de/lexikon/projektplanung.php [Zugriff: 17.03.2018]}   Weiters wird die Projektmanagementmethode und alle, für diese Methode wichtigen, Tools, welche verwendet wurden, erklärt.

\section{Kick-Off-Meeting}

Im Rahmen dieser Diplomarbeit, wurde als Erstes ein Kick-Off-Meeting mit den Projektbetreuern organisiert, bei der die Durchführung besprochen wurde. Anschließend wurden, die zu erledigenden Dokumente, die Umwelt- bzw. Umfeldanalyse und die Risikoanalyse, festgelegt. Zuletzt wurde die Projektmanagementmethode festgelegt, welche für die Diplomarbeit am besten geeignet ist. Nach dem Kick-Off-Meeting wurde der Antrag, welcher die Projektziele, die Projektorganisation, die Umwelt- bzw. Umfeldanalyse, die Risikoanalyse, die Meilensteinliste, die Projektressourcen und externe Kooperationspartner enthält, geschrieben und dem Abteilungsvorstand der Informationstechnologie, Dr. Gerhard Hager, übergeben.

\section{Projektmanagement und Kommunikation}

Für diese Diplomarbeit entschied sich das Team dazu, die Projektmanagementmethode Scrum zu verwenden. Für diese Methode ist ein Tool, welches die Arbeitspakete enthält, notwendig. Hierbei verwendet das Team die Webseite taiga.io, welches mit der Webseite für Zeiterfassung, toggl.com, verwendet werden kann. Für die Dokumentenverwaltung wird Sharepoint verwendet, welche mit den HTL Rennweg Accounts verbunden sind. Dabei hat der Hauptbetreuer, sowie der Nebenbetreuer Zugriff darauf und kann alle Dokumente einsehen.

\subsection[Projektmanagementsmethode]{Projektmanagementmethode\protect\footnote{\label{foot:2}vgl.http://projektmanagement-definitionen.de/glossar/scrum/  [Zugriff: 17.03.2018]}}

Die vom Projektteam gewählte Projektmanagementmethode, ist Scrum. Scrum ist eine agile Projektmanagementmethode und wird seit den 90er Jahren in der Softwareentwicklung verwendet und findet dort seinen Ursprung. Es werden drei Rollen definiert, der Product Owner, das Entwicklungsteam und der Scrum Master. 

\subsubsection[Rollen]{Rollen\protect\footnote{\label{foot:2}vgl.https://www.scrum.de/was-macht-product-owner/ [Zugriff: 17.03.2018]} \protect\footnote{\label{foot:2}vgl.https://www.projektmagazin.de/glossarterm/entwicklungsteam-scrum [Zugriff: 17.03.2018]} \protect\footnote{\label{foot:2}vgl.http://agiles-projektmanagement.org/scrum-rollen/scrum-master/ [Zugriff: 17.03.2018]}}   
\paragraph{Product Owner}
\leavevmode \\
Der Product Owner stellt Anforderungen an das Produkt und priorisiert diese. Dabei handelt es sich immer um eine Einzelperson und keinen Ausschuss. Somit darf diese Rolle auch nicht aufgeteilt werden. Weiters, ist er für das Product Backlog zuständig und sorgt dafür, dass der Wert des Produkts maximiert wird, in dem er die Einträge maximiert und so definiert, dass das Entwicklungsteam die Einträge versteht. Er ist die einzige Person, welche Änderungen vornehmen darf und dafür rechenschaftspflichtig ist. Die Entscheidung, ob ein Sprint vorzeitig beendet werden darf, liegt ebenfalls beim Product Owner. Doch die Hauptaufgabe des Product Owners liegt darin, Rücksprache mit dem Management und Stakeholdern zu halten, um die Arbeit zu optimieren und somit eine Fortschrittskontrolle zur Erreichung des Sprintziels durchzuführen. Im Sprintreview erklärt er, welche Product-Backlog-Einträge fertig sind und welche noch zu erledigen sind. 

\paragraph{Entwicklungsteam}
\leavevmode \\
Alle Mitglieder des Entwicklungsteams sind gleichberechtigt und haben untereinander keine Hierarchie. Somit haben sie keine Weisungsbefugnisse und tragen zur Erstellung des Produktes bei. Mitglieder ohne IT-Kenntnisse werden ebenfalls als Entwickler bezeichnet. Allerdings sollte das Entwicklungsteam eine Größe zwischen drei und neun Personen betragen. Die vom Product Owner spezifizierten Einträge, werden vom Entwicklungsteam umgesetzt, wobei sie die Machbarkeit und den Aufwand der Backlog Einträge beim Sprint Planning beurteilen und über den Umfang des Backlogs entscheiden. Das Entwicklungsteam entwickelt das Produkt und ist für die Abarbeitung des Backlogs zuständig. Allerdings in einer Reihenfolge, welche vom Product Owner festgelegt wird. Am Ende eines jeden Sprints wird ein Prototyp präsentiert und über den erzielten Fortschritt berichtet. 

\paragraph{Scrum Master}
\leavevmode \\
Der Scrum Master ist kein Projektleiter, sondern dafür verantwortlich, dass die Projektmanagementmethode Srum, richtig ausgeführt wird. Falls Scrum neu eingeführt wird, ist es seine Aufgabe und Pflicht, dem restlichen Team den Ablauf zu erklären. Er trifft während des Projektablaufes keine Entscheidungen und darf nicht in Entwicklungsprozess eingreifen. Sein Aufgabenbereich liegt in der Moderation und der Vermittlung an Informationen zwischen Product Owner und dem Entwicklungsteam. Ein Scrum Master entlastet weiters das Projektteam, in dem er zentrale Aufgaben übernimmt und über die Werte und Regeln eines Projektes wacht. Weiters leitet er Meetings, in welchen er unparteiisch zu handeln hat. Somit hat ein Scrum Master Meinungsverschiedenheiten oder generelle Konflikte zu beseitigen. 

\subsubsection[Product Backlog]{Product Backlog\protect\footnote{\label{foot:2}vgl.http://scrum-fibel.de/artefakte/sprint\%20backlog.html [Zugriff: 17.03.2018}}
Der Product Backlog wird vom Product Owner angelgt, gepflegt, geordnet und priorisiert und ist eine Sammlung von Anforderungen. Das Backlog enthält „User-Stories“, welche die Bedürfnisse der Nutzer widerspiegeln. Diese enthalten grobe Anforderungen, die für das Endergebnis wichtig sind und werden beim Sprint-Planning Meeting festgelegt. Die User-Stories müssen in kleinstmögliche Aufgaben aufgegliedert werden, damit das Team diese im Sprint abarbeiten kann. Ein Sprint dauert in der Regel ein bis vier Wochen, diese wird vom Scrum Master festgelegt. Weiters spielt die Priorisierung eine wichtige Rolle, denn diese entscheidet darüber, was im nächsten Sprint abgearbeitet wird. Diese darf während dem Sprint nicht verändert werden. Allerdings werden sie in Absprache mit den Stakeholdern korrigiert und angepasst, sofern die Dringlichkeit und Notwendigkeit eine große Rolle spielen. Um einen Überblick über alle Aufgaben zu behalten, wird ein Sprint Backlog angelegt. 

\subsubsection[Sprint Backlog]{Sprint Backlog\protect\footnote{\label{foot:2}vgl.http://scrum-fibel.de/artefakte/sprint\%20backlog.html [Zugriff: 17.03.2018]}}
In einem Sprint Backlog befindet sich eine Liste von Aufgaben. Diese sind essentiell um die Anforderungen des Product Backlogs umzusetzen. Das Sprint Backlog entsteht aus der Sprint Planung, wobei der Aufwand einer Aufgabe in Stunden geschätzt wird. Die Aufgaben nimmt sich jedes Teammitglied selbst und nutzt den Sprint Backlog zur Verfolgung der Aufgaben. 

\subsubsection{Sprint Planning}


\subsubsection {Daily Scrum}

\subsubsection{Sprint-Review}

\subsubsection{Sprint-Retrospective}

\subsubsection {Product Increment}

\subsection{Zeiterfassung}
\subsection{Kommunikationsmittel}
\subsection{Dokumentenverwaltung durch Sharepoint}
