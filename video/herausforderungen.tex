\section{Herausforderungen}
\subsection{Interview mit dem Abteilungsvorstand}
Das Problem bei dem Dreh, mit dem Interview des Abteilungsvorstandes, war, dass die Blende zu weit geöffnet war, und so es nicht möglich war die Schärfentiefe zu erzeugen. Die Schärfentiefe wird durch die Blende, der Sensorgröße und der Brennweite definiert. Da man an der Sensorgröße nach dem Kauf der Kamera nichts mehr ändern kann, muss man sich mit den anderen zwei Bereichen auseinandersetzen. Je weiter die Blende geschlossen ist, desto mehr Schärfentiefe kann erzielt werden. Je offener die Blende ist, umso weniger Schärfentiefe hat man. Die Brennweite beeinflusst die Schärfentiefe insofern, da je größer die Brennweite ist, also je näher man an ein Objekt heranzoomt, desto unschärfer wird der Hintergrund. 
Wie man in der Abbildung sehen kann, befinden sich im Hintergrund zu viele Störfaktoren, die aufgrund der fehlenden Schärfentiefe gut erkennbar sind, und dem Zuseher vom eigentlichen Bild ablenken würden. 
Das Programm Adobe Premiere Pro ermöglicht es mittels einer Maske einen Maskenpfad zu setzen, den man später auch "bewegen" kann. Das bedeutet, dass sich die Maske Frame für Frame fortbewegt und sich anpasst. Jedoch war dies hier nur begrenzt möglich, da sich der Hintergrund nicht sonderlich abhebt, und so die Maske sich nicht mitbewegen konnte. Der Maskenpfad musste schließlich bei manchen Frames manuell nachbearbeitet werden.
Man kann beim Erstellen der Maske entscheiden, ob man die Maske mittels einer Ellipsen Form, einer Rechteck Form oder mit einem Zeichenstift erzeugen will. Bevor man die Maske zeichnet, wählt man den Effekt aus, mit dem man arbeiten will. Im folgenden Fall, war es den Hintergrund unscharf zu zeichnen, also war die Verwendung eines Weichzeichners nötig. Man zieht den gewünschten Effekt ins Schnittfenster und anschließend kann man das gewünschte Werkzeug auswählen, in dem Fall den Zeichenstift. Dies war deswegen notwendig, da man auch die Person auswählen musste, was mit einem Kreis oder Rechteck nicht möglich gewesen wäre. Hat man ein Werkzeug ausgewählt, kann man auch schon die Maske setzen. Adobe Premiere Pro bietet zusätzlich noch eine weiche Kante auf Masken anzuwenden. Dies ist deswegen von großer Bedeutung, da ansonsten der maskierte Bereich mit dem anderen nicht "verschmilzen" würde. Das heißt, man würde die harten Kanten zwischen den zwei Objekten deutlich erkennen. Die weiche Kante glättet den Maskenauswahlrahmen und ermöglicht einen schönen Übergang zwischen Maskenkante und dem anderen Bereich.
\subsection{Tag der offenen Tür Video}
Das Video des Tag der offenen Tür wurde am Gang in der Schule gedreht. Da unter anderem manche Lichter immer wieder aus und an gingen, waren oftmals andere Lichtverhältnisse vorhanden. Diese mussten in der Post Production ebenfalls nachbearbeitet werden. Bei der Bearbeitung der Farbe wurden die Lumetri Scopes Grafiken herangezogen. Diese dienten insofern gut, da man auf einen Blick gesehen hat, von welcher Farbe am meisten Information vertreten ist, und wo noch etwaige Farbinformation fehlt. Mithilfe des Vektorskops, Histogramms und des Waveform-Monitors konnte die Farbe korrigiert werden.
\paragraph{Vektorskop}
\leavevmode \\
"Das Vektorskop gibt Auskunft über die Farbverteilung im Bild."\footnote{\label{}buch} Auf dem Vektorskop sind die Grundfarben aus dem RGB und dem CMYK Farbraum abgebildet. Also: Rot, Grün, Blau, Cyan, Magenta, Yellow. Je weiter der Signalpunkt sich vom Zentrum entfernt, desto höher ist die jeweilige Farbsättigung. Hätte man ein Schwarz-Weiß-Bild würde man einen Punkt in der Mitte sehen.
\paragraph{Histogramm}
\leavevmode \\
Das Histogramm zeigt die Häufigkeitsverteilung von Helligkeit und Farbe, die im Bild vorkommt. Weiters zeigt das Histogramm, ob das Bild korrekt belichtet wurde. Das Histogramm geht von 0,0,0 für RGB Schwarz bis 255,255,255 für RGB Weiß. Die Schatten befinden sich beim Histogramm unten, also in Richtung 0,0,0, wobei man die Lichter oben, also in Richtung 255,255,255 findet.
\paragraph{Waveform-Monitor}
\leavevmode \\
Der Waveform-Monitor zeigt die Helligkeitsverteilung im Bild. Die Helligkeit wird als Häufigkeitsverteilung wiedergegeben, wobei 100\% Weiß einen Messwert von 100 IRE entsprechen. "IRE ist die international übliche Skalierung. 0 IRE entsprechen daher Schwarz. Mithilfe des Waveform-Monitors kann man einfach die sogenannten Lichter, Mitten und Tiefen einfach analysieren. 
\subsection{Video mit einem Absolvent}
Da beim Dreh des Videos der Schüler der zweiten Klasse etwas unschärfer war, war es nötig manuell nachzuschärfen. "Mit dem Effekt 'Scharfzeichner' kann der Kontrast an den Stellen mit Farbänderungen verstärkt werden." Jedoch sollte der Effekt sparsam eingesetzt werden, da es schnell unnatürlich wirken kann. Bei dem Video gab es nicht grobe Herausforderungen. Bei dem Video wurden jedoch verstärkt verschiedene Blenden eingesetzt, um es spannend und abwechslungsreich zu gestalten. 
\paragraph{Effektblenden}
\leavevmode \\
Die Effektblende wird als Gestaltungsmittel bezeichnet, mit der man einen Übergang von einem Ereignis zum anderen besser bildlich darstellen kann. Bei dem Quiz mit dem Absolventen und dem Schüler wurde einerseits die Effektblende "Wegschieben", als auch "Einzoomen \& Auszoomen" und "Filmblende" verwendet. Die Blende "Wegschieben" dient dazu von einem Bild,  also zum Beispiel von einer Antwort zur nächsten Frage überzugehen. Der "Einzoomen \& Auszoomen" Effekt wurde beim Einblenden der Antworten verwendet. Zuletzt wurde die "Filmblende" für das Einblenden der Punkte verwendet.
